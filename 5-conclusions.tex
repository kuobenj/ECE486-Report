% This file is part of the multi-tex ECE 486 Final Project Report 
% file name: 5-conclusions.tex

% YOU DO NOT COMPILE FROM THIS SINGLE FILE, read the following

% included files:
% (Makefile)
% Makefile -- run make Makefile (on Mac or Linux), it takes care of LaTeX compilation

% (*.PDF)
% report.pdf -- report file, print it out and submit it

% (*.TEX)
% report.tex -- main file, pdflatex this file (tested on Mac and Linux)
% 0-title-page.tex -- title page 
% 1-introduction.tex -- chapter 1 
% 2-mathematical-model.tex -- chapter 1 (lagrange equations of motion) and chapter 4 (linearisation)
% 3-full-state-feedback-control-friction-compensation.tex -- chapter 2 and chapter 4 (two state and three state feedback controller design) 
% 4-full-state-feedback-control-decoupled-observer.tex -- chapter 4 (controller design) and chapter 5 (observer design for estimated state)
% 5-conclusions.tex -- (this file) conclusion
% 6-extra-credit.tex -- (optional) add thsese pages if you have demoed chapter 6 and chapter 7

\section{Conclusions}
Both controllers have advantages and disadvantages over the other. In terms of robustness, the full-state feedback controller performs better over the observer design. This is shown by the robustness comparison table. The full-state feedback design can handle larger pulse and constant disturbance than the observer design. The full-state design has access to all the states of the system while the observer design could only estimate the states, hence, the full-state design will have higher precision to the states of the system and hence be more robust as compared to the observer design.